\section{Quantum Computing}
As Deutsch pointed out in the original paper which attempted to define a quantum Turing machine, computation is a fundamentally physical notion. When I type numbers into my calculator and press a button, I am declaring the rules for some initial configuration of electrons in space, and when I press the button which initiates a computation, that initial system is set in motion for a span of time, and the final configuration of the system has some correspondence that I can go by to interpret a result. If a process is computable, then it can be implemented in physical reality, but what would, or as perhaps \textit{should}, tie the notions together more tightly and pull our abstract mathematical notion of computation into the realm of physics, is the converse. Wouldn't the reader agree? Shouldn't it be the case that any process implementable in physical reality should, up to one's interpretation, correspond to a computation? 
\subsection{Reversibility}
The core tenant of quantum mechanics and modern physics is that at it's most basic level, reality evolves linearly, and even more so that it evolves in a way which is reversible in the simplest sense. If all physical processes are reversible, then all computations should be as well. We begin by observing this phenomena with the Turing machine that we are used to. 
\begin{definition}
	A \textbf{reversible} Turing machine is a deterministic Turing machine for which each configuration has at most one predecessor. 
\end{definition}
Thus a Turing machine is reversible if we can look at it's current configuration, and have it be unambiguous what configurations the machine used to be in, all the way up to it's initial setup. Note that we say \textit{at most} one predecessor, to emphasize that a configuration can easily have none. The surprising result here will be that requiring there to be at most one actually necessitates that there be one! The following theorem gives us this as a corollary, but unintentionally. The intention right now is to have a set of necessary and sufficient conditions to determine if a Turing machine is reversible:
\begin{theorem}
	A Turing machine $M$ is reversible iff the following two conditions hold:
	\begin{itemize}
	\item[1.] Each state of $M$ can be entered while moving in only one direction. I.e. if $\delta(p_1,\sigma_1) = (\tau_1,q,d_1)$ and $\delta(p_2,\sigma_2) = (\tau_2,q,d_2)$, then $d_1 = d_2$.
	\item[2.] The transition function $\delta$ is one-to-one when direction is ignored. 
	\end{itemize}
\end{theorem}
\begin{proof}
	Suppose $M = (\Sigma,Q,\delta)$ is a TM satisfying these two conditions. Consider an arbitrary configuration $c = (T,q,z)$. Since the state $q$ can only be entered from one direction, this determines exactly where the tape head had to be in a hypothetical previous step, determining that any configuration $c'$ which yielded $c$ must have had cursor position $z'$. Now, consider the tape symbol at this previous cursor position, $\lambda = T[z']$. Suppose there exists a symbol-state pair $(\lambda',q')$ such that (ignoring the direction output $\delta(\lambda',q') = (\lambda,q)$. Since $\delta$ is one-to-one barring direction, this pair is unique to have this property. But then the only configuration $c'$ which could possibly yield $c$ is the one with tape configuration $T' = T$ except for $T'[z'] = \lambda'$, state $q'$, and cursor position $z'$. Thus arbitrary configurations have at most one predecessor, so $M$ is reversible.
	\par Now suppose that a Turing machine $M$ is reversible. We first prove the necessity of property $1$. Suppose that $\delta(p_1,\sigma_1) = (\tau_1,q,L)$, and $\delta(p_2,\sigma_2) = (\tau_2,q,R)$, i.e. we have two different state-symbol pairs which yield the same configuration $q$. We can then construct two configurations which lead to the same following configuration: Let $c_1$ be a configuration where the machine is in state $p_1$ reading symbol $\sigma_1$, and the symbol two tape cells left is $\tau_2$, and let $c_2$ be the configuration which is identical to $c_1$ except that $\sigma_1$ is replaced with $\tau_1$, $\tau_2$ is replaced with $\sigma_2$, and the cursor is pointed two tape cells left of $c_1$ in state $p_2$. It is clear from drawing out these configurations and inspecting the resulting configuration that the resulting configuration yielded by both $c_1$ and $c_2$ is the same. (If someone is reading this and wants to render it for me, I'd appreciarte it.) This contradicts $M$ being reversible, so it follows that states can only be entered from at most one direction. 
	\par Finally, continuing to assume that $M$ is reversible, we prove the necessity of property $2$. Suppose $\delta(p_1,\sigma_1) = \delta(p_2,\sigma_2)$, with $(p_1,\sigma_1) \neq (p_2,\sigma_2)$. Then any two configurations which differ only in the state and symbol under the tape head will yield the same configuration in one step, and so clearly $M$ is not reversible. Thus for $M$ to be reversible, $\delta$ must be one-to-one, up to direction. 
\end{proof}
\begin{corollary}
	A Turing machine $M$ is reversible iff every configuration of $M$ has \textit{exactly} one predecessor. 
\end{corollary}
\begin{proof}
	Obviously if $M = (\Sigma,Q,\delta)$ is a Turing machine in which every configuration has exactly one predecessor, then $M$ is reversible. Conversely, let $M$ be reversible, i.e. we have the (supposedly) weaker condition that every configuration has \textit{at most} one predecessor. Let $c = (T,q,z)$ be an arbitrary configuration of $M$. Now the key observation is to note that the domain of the transition function is size $|\Sigma||Q|$, while the size of the codomain is $2|\Sigma||Q|$. Every state-symbol input has to map to \textit{something}, and by the above result, each potential state-symbol output pair can only be mapped to a single time. Furthermore, since each state can only be entered from a single direction, the effective size of the codomain is halved to that of the domain, meaning that every unique state-symbol pair \textit{has to} be an output of $\delta$, simply by the pigeonhole principle. This is all to say, there must exist a state/symbol pair $(\lambda,q')$ such that the state of $\delta(\lambda,q')$ is $q$. Let the direction of this output be $d_q$, and let $\bar{d_q}$ be the reverse direction. Then the configuration $c'$ whose tape is identical to $T$ except that the symbol adjacent to $z$ in the $\bar{d_q}$ direction is $\lambda$, whose state is $q'$, and whose tape head is pointed at that $\lambda$, clearly yields $c$ in one step. Thus, we have constructed a configuration which yields $c$, and since $M$ was reversible, this is unique, completing the proof. 
\end{proof}
The significance of this result follows from the following initially very odd way of viewing a Turing machine's operation. Let $M$ be a Turing machine. 	

