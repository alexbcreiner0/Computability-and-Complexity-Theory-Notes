\subsection{Query Complexity}
\begin{definition}
	Let $\tau$ be a vocabulary. Define $STRUC[\tau]$ to be the set of all finite structures appropriate to $\tau$. If a vocabulary has no function symbols, then we call it a \textbf{relational vocabulary}. Unless otherwise noted in this section, all vocabularies will be relational. A \textbf{relational database} is a finite structure appropriate to a relational vocabulary.
\end{definition}
\begin{definition}
	A \textbf{query} is a polynomial bounded mapping $I: STRUC[\sigma] \to STRUC[\tau]$. I.e. it takes structures of one vocabulary to structures of some other vocabulary, and there is a polynomial $p$ such that for all $\mathcal{A} \in STRUC[\sigma]$, $|I(\mathcal{A}| \leq p(|\mathcal{A}|)$. A \textbf{boolean query} is a map $I_b: STRUC[\sigma] \to \{0,1\}$. A boolean query may also be thought of as a subset of $STRUC[\sigma]$ - namely set of structures $\mathcal{A}$ such that $I_b(\mathcal{A}) = 1$. 
For the remainder of the section, we employ some specialized notation. First, if $\mathcal{A}$ is a structure over some vocabulary, then $|\mathcal{A}|$ will denote the universe of $\mathcal{A}$ rather than the cardinality, while $||\mathcal{A}||$ will denote the cardinality.
\end{definition}
Note that for any first order sentence $\phi$ over a vocabulary $\tau$ naturally defines a boolean query $I_{\phi}$ over $STRUC[\tau]$: namely, $I_{\phi}(\mathcal{A}) = 1 \iff A \models \phi$. Note the implicit decision problem here: given an input, a structure $\mathcal{A}$, is it the case that $\mathcal{A} \models \phi$? 
\par Our definition of query needs to be refined. We want the most natural definition but this requires some technicality.
\begin{definition}
	Let $\sigma$, $\tau$ be vocabularies, where $\tau = \langle R_1^{a_1},...,R_k^{a_k},c_1,...,c_s \rangle$ and $k \in \mathbb{N}$. Define the query $I: STRUC[\sigma] \to STRUC[\tau]$ as follows. It is specified by an $r+s+1$-tuple of expressions $\phi_0,\phi_1,...,\phi_r,\psi_1,...,\psi_s$ over $\sigma$, in the following way. The expression $\phi_0$ specifies the universe of $I(\mathcal{A})$, while $\phi_1,...,\phi_r$ specify the relations of $I(\mathcal{A}$, and $\psi_1,...,\psi_s$ specify the constants of $I(\mathcal{A})$. Specifically, 
	\[ |I(\mathcal{A})| = \{\langle b^1,...,b^k \rangle : \mathcal{A} \models \phi_0(b^1,...,b^k)\} \]
And more often than not, $phi_0$ will simply be taken to be a tautology, so that $|I(\mathcal{A})| = |\mathcal{A}|^k$. Next, the relation $R_i^{I(\mathcal{A})} \subseteq |I(\mathcal{A})|^{a_i}$ is defined by
	\[ R_i^{I(\mathcal{A})} = \{ (\langle b_1^1,...,b_1^k \rangle,...,\langle b_{a_i}^1,...,b_{a_i}^k \rangle ) \in |I(\mathcal{A})|: \mathcal{A} \models \phi_i(b_1^1,...,b_{a_i}^k) \} \]  
	(Note then that implicitly $\phi_i$ defines a relation on $\mathcal{A}^{ka_i}$.) 
	Each constant symbol $c_j^{I(\mathcal{A})}$ is an element of $|I(\mathcal{A})|$, namely it is defined to be the unique $\langle b^1,...,b^k \rangle$ such that $\mathcal{A} \models \psi_j(b^1,...,b^k)$. 
	\par When we need to be formal, we let $a = \max\{a_i : 1 \leq i \leq r\}$ and let the free variables of $\phi_i$ be $x_1^1,...,x_1^k,...,x_{a_i}^1,...,x_{a_i}^k$, with the free variables of $\phi_0$ and the $\psi_j$'s be $x_1^1,...,x_1^k$. 
	\par If the formulas $\psi_j$ all have the property that for all $\mathcal{A} \in STRUC[\sigma]$, 
	\[ |\{ \langle b^1,...,b^k \rangle : (\mathcal{A},i) \models \phi_0 \wedge \psi_j \} | = 1 \]
	where $i(x_1^1) = b^1,...,i(x_1^k) = b^k$, then we call $I$ a \textbf{$k$-ary first order query}, and write $I = \lambda_{x_1^1,...,x_a^k}\langle \phi_0,...,\psi_s \rangle$. A \textbf{first order query} is either a boolean query, or a $k$-ary query for some $k$. We define \textbf{FO} to be the set of all first order boolean queries. Let \textbf{Q(FO)} be the set of all first-order queries. 
\end{definition}

